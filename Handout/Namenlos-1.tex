%% start of file `template.tex'.
%% Copyright 2006-2013 Xavier Danaux (xdanaux@gmail.com).
%
% This work may be distributed and/or modified under the
% conditions of the LaTeX Project Public License version 1.3c,
% available at http://www.latex-project.org/lppl/.


\documentclass[11pt,a4paper,sans]{moderncv}        % possible options include font size ('10pt', '11pt' and '12pt'), paper size ('a4paper', 'letterpaper', 'a5paper', 'legalpaper', 'executivepaper' and 'landscape') and font family ('sans' and 'roman')

% modern themes
\moderncvstyle{banking}                            % style options are 'casual' (default), 'classic', 'oldstyle' and 'banking'
\moderncvcolor{blue}                                % color options 'blue' (default), 'orange', 'green', 'red', 'purple', 'grey' and 'black'
%\renewcommand{\familydefault}{\sfdefault}         % to set the default font; use '\sfdefault' for the default sans serif font, '\rmdefault' for the default roman one, or any tex font name
%\nopagenumbers{}                                  % uncomment to suppress automatic page numbering for CVs longer than one page

% character encoding
\usepackage[utf8]{inputenc}                       % if you are not using xelatex ou lualatex, replace by the encoding you are using
%\usepackage{CJKutf8}                              % if you need to use CJK to typeset your resume in Chinese, Japanese or Korean

% adjust the page margins
\usepackage[scale=0.75]{geometry}
%\setlength{\hintscolumnwidth}{3cm}                % if you want to change the width of the column with the dates
%\setlength{\makecvtitlenamewidth}{10cm}           % for the 'classic' style, if you want to force the width allocated to your name and avoid line breaks. be careful though, the length is normally calculated to avoid any overlap with your personal info; use this at your own typographical risks...

\usepackage{import}

% personal data
\name{GFS}{Informatik}
\title{Versionskontrolle}                               % optional, remove / comment the line if not wanted
\address{Liam Wachter, TGI 13/2}{}{}% optional, remove / comment the line if not wanted; the "postcode 
\homepage{github.com/sevenmaster/InformatikGFS Dieses Handout in digital und bunt}                         % optional, remove / comment the line if not wanted
%\extrainfo{additional information}                 % optional, remove / comment the line if not wanted
%\photo[64pt][0.4pt]{picture}                       % optional, remove / comment the line if not wanted; '64pt' is the height the picture must be resized to, 0.4pt is the thickness of the frame around it (put it to 0pt for no frame) and 'picture' is the name of the picture file
%\quote{Some quote}                                 % optional, remove / comment the line if not wanted

% to show numerical labels in the bibliography (default is to show no labels); only useful if you make citations in your resume
%\makeatletter
%\renewcommand*{\bibliographyitemlabel}{\@biblabel{\arabic{enumiv}}}
%\makeatother
%\renewcommand*{\bibliographyitemlabel}{[\arabic{enumiv}]}% CONSIDER REPLACING THE ABOVE BY THIS

% bibliography with mutiple entries
%\usepackage{multibib}
%\newcites{book,misc}{{Books},{Others}}
%----------------------------------------------------------------------------------
%            content
%----------------------------------------------------------------------------------
\begin{document}
%\begin{CJK*}{UTF8}{gbsn}                          % to typeset your resume in Chinese using CJK
%-----       resume       ---------------------------------------------------------
\makecvtitle

\small{Versionskontrollsysteme (auch Versionsverwaltungssysteme) sind Systeme zur Erfassung und Verwaltung von Änderungen einzelner Dateien oder ganzer Projekte. Diese ermöglichen es nicht nur jeder Zeit jede erfasste Version wiederherzustellen zu können, sondern auch mehrere Versionen parallel zu entwickeln und zusammenführen zu können. Außerdem können durch Versionskontrollsysteme mehrere Entwickler gleichzeitig an einem Projekt arbeiten.}

\section{Grundlegende Konzepte}

\vspace{6pt}

\begin{itemize}

\item{\cventry{}{git fetch; git merge; git push}{copy - modify - merge}{}{}{\vspace{3pt}Damit mehrere Entwickler gleichzeitig an der gleichen Datei arbeiten können wird kopiert sich jeder Entwickler die aktuelle Version und macht Änderungen daran. Die Änderungen werden später zusammengeführt (gemerged). Das geschieht mit einem Merge-Algorithmus. Kann dieser die Änderungen nicht zusammenführen, da sie im Widerspruch zueinander stehen, spricht man von einem Mergekonflikt. Diese müssen von Menschen aufgelöst werden.}}

\vspace{6pt}

\item{\cventry{}{git commit}{Commit}{}{}{\vspace{3pt}Mit einem Commit speichert man die aktuelle Version von Dateien oder Verzeichnisse. Ein Commit einhält nicht nur eine Referenz auf den eingepflegten Zustand des Projekts, sondern enthält auch Autor, eine Beschreibung der vorgenommenen Änderungen, sowie den Namen des vorangegangenen Commit. Commits sind, wie alle Git-Objekte, mit ihrer SHA1 eindeutig benannt.}}

\vspace{6pt}

\item{\cventry{}{git branch; git checkout}{branching}{}{}{\vspace{3pt}Mit einem Commit speichert man die aktuelle Version von Dateien oder Verzeichnisse. Ein Commit einhält nicht nur eine Referenz auf den eingepflegten Zustand des Projekts, sondern enthält auch Autor, eine Beschreibung der vorgenommenen Änderungen, sowie den Namen des vorangegangenen Commit. Commits sind, wie alle Git-Objekte, mit ihrer SHA1 eindeutig benannt.}}

\end{itemize}


% Publications from a BibTeX file without multibib
%  for numerical labels: \renewcommand{\bibliographyitemlabel}{\@biblabel{\arabic{enumiv}}}% CONSIDER MERGING WITH PREAMBLE PART
%  to redefine the heading string ("Publications"): \renewcommand{\refname}{Articles}
\nocite{*}
\bibliographystyle{plain}
\bibliography{publications}                        % 'publications' is the name of a BibTeX file

% Publications from a BibTeX file using the multibib package
%\section{Publications}
%\nocitebook{book1,book2}
%\bibliographystylebook{plain}
%\bibliographybook{publications}                   % 'publications' is the name of a BibTeX file
%\nocitemisc{misc1,misc2,misc3}
%\bibliographystylemisc{plain}
%\bibliographymisc{publications}                   % 'publications' is the name of a BibTeX file

%-----       letter       ---------------------------------------------------------

\end{document}


%% end of file `template.tex'.
